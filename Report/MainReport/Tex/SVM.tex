\section{Support Vector Machines}

Support Vectors Machines have become a well established tool within machine learning. They work well in practice and have now been used across a wide range of applications from recognizing hand-written digits, to face identification, text categorisation, bioinformatics, and database marketing.

An SVM is an abstract learning machine which will learn from a training data set and attempt to generalize and make correct predictions on novel data.

Since the problem outlined in this project is a binary classification problem. SVMs are an applicable method.

For the training data we have a set of input vectors, denoted \(x_i\) , with each input vector having a number of component features. These input vectors are paired with corresponding labels, which we denote \(y_i\) and there are \(m\) such pairs (i = 1, ... , m).

For this project. It is considered that an oncogenic (cancer causing) sample would be \(y_i=+1\), benign sample would \(y_i=-1\) and the matching \(x_i\) are input vectors encoding various genomic features derived from each mutation \(i\)